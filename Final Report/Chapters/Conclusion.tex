\chapter{Conclusion and Possible Future Work}

This project set out to develop an application to help German learners discover and read German language articles. The application developed in the process achieved this, as most of the users who tested it thought that it helped them. However, the bugs in the application left it as a sub-par experience, with half the testers not interested in a commercial application. 

The application was designed and coded in a logical, structured way. Although more advanced techniques that used machine learning algorithms were not used, they were considered during the development cycle but were disregarded due to the data required for them to be functional not being available.

Developing the application further, in the hopes of launching commercially, would probably require a restructuring  of the project into something similar to the idea that was expressed in the evaluation. The design of the gloss entries as well as application as a whole would most likely need professional input as the current design is something that was put together using various elements from pre existing CSS libraries and not really something for a professional application.

The translation software of the application should be replaced as most of the bugs in the application were caused by that, so a large amount of source code would have to be rewritten to interface the application with the new translation software.

There are various legal problems that a commercial version of the application would have to deal with, particularly the reuse of the article content. A commercial application would have to make sure that the articles are licensed from their various sources as the articles, being recent, are all still in copyright.

Performing all of these actions for a commercial launch of the application would be a massive undertaking and not one worth doing at the time of writing due to financial constraints. Even if it were, the lack of intent from the testers to use the application suggests a commercial launch would likely require even more work. At some point in the future, a commercial release may be a viable option, but at that point rewriting the entire code base will probably be more feasible due to potential new technologies. 