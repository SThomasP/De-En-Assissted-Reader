\chapter{User Testing and Feedback}

Once the application had been developed, it was then user tested. The application was used by German learners of various experience and they then gave feedback on the application. This chapter documents how the application was tested and then proceeds to analyse the feedback from the testing, examining the successes and failures of the application.

\section{Testing Methodology}

In order to carry out this testing, University of Southampton policy required that the testing had to have ethics approval from the Faculty of Physical Sciences and Engineering's ethics committee. This was applied for and approved through the ERGO system, with a submission ID number of 40080, being classed as a category C research project. While getting this approval took longer than expected, it was approved in good time. Copies of all the documents that were submitted to ERGO are available in the design archive. 

The plan for testing was that the user would test the application by using it to find and read three articles, they would then go on to fill out a short survey on their experiences with it. This survey asked the users the following things. 
\begin{itemize}
	\item Whether or not the application helped them understand the articles.
	
	\item Whether or not the application helped them with words in they article that they did not know.
	
	\item Whether or not the articles were of a suitable difficulty level.
	
	\item Whether or not the translations were accurate.
	
	\item Whether or not the application was easy to use.
	
	\item Whether or not the user liked the position of the gloss annotations.
	
	\item Whether or not the user would use the application if available. 
	
	\item A comment box for any additional thoughts the user may have. 
\end{itemize}


In addition the user's use of the application was logged, so that this information could then be compared to their survey results. Doing this provided a more complete picture of the user's experience with the application, allowing for a more in-depth analysis of their results. 

\section{User Feedback and Analysis}

There were four people who tested the application. Of the four, two rated themselves beginner, one intermediate and one advanced. This was a decent range of experience levels, although more intermediate and advanced level users would have been appreciated, as this was the target demographic that the application was designed for.

\subsection{Overall Feedback for the Application}

Feedback for the application as a whole was mixed, with half the users saying that they would not use the application if it was available. One of them commented that they would prefer to use the application as a browser extension, looking up definitions for words that they discovered while browsing the web.

There was disagreement in opinion on whether or not the application was easy to use. Half of the users thought that it was, while another was neutral on the subject and the final one disagreed with the statement, presumably finding it difficult to use. This might be solvable by improving the design of the application, providing more detailed instruction on how to use the application as well as making the existence of the gloss system more obvious.


\subsection{The Article Discovery and Difficulty Rating System}

The advanced and intermediate level users said that the articles they read were of an appropriate level for them. With the advanced level one noting that they found the articles with higher difficulty ratings to be of a higher difficulty. This suggests that the level mapping system is functional in its intended use.  

A problem that occurred with one of the beginner level users, where the articles listed were mostly rated as "hard" or "very hard." This lead to a problem with the user not being able find articles of a suitable difficulty level. This problem could potentially be solved by pulling from different news sources based on the users experience level. For example, a beginner level user would get articles from sources that specifically target children or learners, while an advanced level user would get articles from a general site.

\subsection{The Gloss and Article Reader}

There was mostly positive results on whether or not the gloss helped increase the users' understanding of the articles. Mostly they thought that it had, although for some articles they thought that it had not. There is not enough data when comparing users and the articles they read to reach a conclusion as to why the gloss was not helpful in those cases. It would require multiple users reading the same article, and the nature of the testing with four users reading from different categories at different times meant that none on the users encountered any of the same articles. 

There was very mixed feedback on whether or not the application helped with understanding of unknown words. The correlation in this appears to come down to the user's experience level, with the advanced user reporting that it did not help at all, while the beginners and the intermediate ones reporting that it helped. This is probably to do with the problems faced in terms of gloss content, which are discussed in detail below.

All the users liked the position of the annotations, meaning that a marginal gloss was probably the best idea in terms of gloss position. However, one of them notes that it would be better if the annotations scrolled automatically to new ones when they appear. Expanding on this idea, it might be possible to use this as a method to dismiss the gloss items automatically as well, dismissing them when they are moved off the top of the screen.

\subsection{Gloss Content}

The most common problem found with the gloss section of the application was the fact that several words produced no usable gloss item. The words missing ranged in scope, from longer compound words that are not commonly found in dictionaries, to more common, smaller words that really should be in the dictionary. However, as this was a third party dictionary API, adding these more common words to the dictionary is not really possible. The solution is to replace the method by which translations and grammatical information are obtained. For words where translations were provided, these were considered by the users to be accurate. 

Returning to the inability for compound words to be translated, there were discussions during meetings with the project supervisor about methods that could be used  to break these words into their component parts and translate those. Adding in this functionality was dropped due to it being considered low priority, but due to the fact that the advanced and intermediate level users both attempted to gloss these words more often than the non-compound words that were in the dictionary, developing it would have most likely been beneficial for the application overall. 

In addition, a user noted several times that the content of the grammatical information provided in the gloss was wrong. The intent behind the design of the gloss was to provide every possible use case of the word in that form, but the user in question found this confusing. It would have been better to try and develop a way to determine from context the exact reason why the word was in that form, providing that as a single use case.  

Overall feedback relating to this was that the dictionary content was lacking. Future builds of the project should probably implement a better system for obtaining translations and grammatical information, as the service that was used in the current application lead overall to more incorrect grammatical information, missing translations and other mistakes than successful gloss attempts. 

A few problems relating to the design of the gloss were also found in the application. Most notably was the fact that gloss items for words over a certain length are impossible to dismiss without zooming out in the web page. Another problem that was noted by one of the users, was that the green colour of the noun glosses and the red colour of other glosses implied success and failure respectively, none of the other users commented on this but the colour scheme of the gloss is easily adjustable.


