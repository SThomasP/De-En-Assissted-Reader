\chapter{User Testing and Feedback}

User testing was done once all six sprints have been completed, but the preparation process started a while before that, the plan was always to have users' test the application by finding and reading three articles, then go on to fill out a short feedback form on their experiences with it. The user's actions in the application would all be logged, in order provided a level of insight for the developer. 

\section{Testing Preparation}

In order to carry out this testing, University of Southampton policy required that the testing had ethics approval from the ECS ethics committee, it was applied for and approved through the ergo system, with a submission ID of 40080, being classed as a category C research project. While getting this approval took longer than the developer expected. It was approved in good time for when the testing was eventually done.

The application was modified slightly in order facilitate testing. The number of articles a user could read was limited to three, once they had read these three articles, they were given a link to the survey.

\section{Testing Results and Feedback Given}

There were four people who tested the application, of the four, two rated themselves beginner, one intermediate and one advanced. This was a decent range of experience levels, although more intermediate and advanced level users would have been appreciated.


\subsection{The Gloss and Article Reader}

The most common problem found with the gloss section of the application was the fact that several words were missing from the dictionary. The words missing ranged in scope, from longer compound words that are not commonly found in dictionaries, to more common, smaller words that really should be in the dictionary. However, as this was a third party dictionary api, adding these more common words to the dictionary is not really possible, the solution should probably be to find a better dictionary api.

Going back to inability for compound words to be translated, there was brief talk during meetings with the supervisor meetings about using machine learning algorithms to break them into component parts and translate those, adding in this logic was dropped due to time constraints. However, due to the fact that the advanced and intermediate testers both attempted to gloss these words more often than the shorter words that were in the dictionary, developing it would have most likely been beneficial for the application overall. 

In addition, a tester noted several times that the content of the grammatical information provided in the gloss was wrong. The intent behind the design of the gloss was to provide every possible reason why a word might currently be in the form that is was in, however the tester found this confusing. It would be better going forward to try and develop a way to determine from context the exact reason the word is currently in that form. 

Overall feedback relating to this was that the dictionary content was lacking, future builds of the project should probably move on to a better system obtaining translations and grammatical information, as the service been used in the current application lead overall to more incorrect grammatical information, missing translations and other mistakes than successful gloss attempts.

A few problems relating to the design of the gloss were also found in the application, Primarily, the fact that long would make the cross symbol used to dismiss a gloss item disappear off the side of the gloss, making it impossible to dismiss the gloss item without zooming out. Another problem that was noted by one of the user, was that the green colour of the noun glosses and the red colour of the other gloss implied success and failure respectively, none of the other testers noted this but the colour scheme of the gloss is easily adjustable.


