\chapter{Evaluation of the Project}

\section{Development}

Development of the project was mostly successful, and the first part of the project proceeded as planned. Development of the article display view and gloss were completed before the Christmas holidays, as well as the progress report detailing the state of the project at the time. Continuing as planned, the paperwork needed for ethics approval was completed over the Christmas holidays. 

Once the Christmas holidays were over, the fact that user testing had to have happened before the Easter holidays and the fact that the main part of the application had already been developed, meant that the development of the project went off the planned schedule. The article discovery system was scaled back several times to meet various constraints. The original idea for this was to build a recommender system that would suggest articles based on a combination of difficulty and category using machine learning algorithms. The main problem with this idea, as well all other ideas that used machine learning algorithms was primarily the fact that there would not be enough data gathered before user testing for these systems to work correctly, so the application was scaled back as these grand ideas were not implementable.

User testing was planned to have more than four testers and the fact that only those four could be found was disappointing. The testing period was advertised to German learners of suitable experience levels but only one of those people turned up. The other testers were then recruited through other means. 

The survey questions as well could have been done better, they reflected an application that was expected to have a recommender system and then they were hastily edited to be more in line with the application as was shown to testers. The decision to have them only read three articles then fill out unique questions about each of those articles could have been replaced with no article limit, and then having survey questions about the application in general, aggregating the results of all articles.

The method of connecting survey results to usage data was very badly done, having the user be given an id number by the application and then having them enter it into the survey. This could have definitely been improved as the logging data was barely used in the actual analysis, instead having the user simply entering their experience level again in the survey would have been a better option. 

\section{The Application}

The application started development incorrectly. Rather than developing it as a web application, the application should have, as one of the testers suggested, been developed as a browser extension. Having the user come across unknown words while browsing German websites and then selecting them and looking them up in the application would have most likely been a better application. This application could still have suggested a number of articles as starting points and then just redirected the user to the article on the site. 

The decision after the initial gloss was developed to build the article discovery system was also a mistake, more effort should have been put into developing advanced parsing, better translations and contextually determining both the grammatical features of the word as well as more accurate translations of the word. A more accurate, contextual gloss would have likely been a better product than the generic gloss and discovery system that was developed.

The criticisms of the application described above are a result of developing an application with the wrong goals. If the application is considered in terms of whether or not it meets its goals, regardless of the of the quality of those goals, then the application is a success.

The article discovery and rating system works as intended, it fulfils its goals as the difficulty ratings were considered accurate and the articles shown were from the category that the user had selected, as such they were relevant to the user.  

The gloss system was also a functional implementation that fulfilled its goals. Any word in the article could be glossed by the user, and the application would respond with translations and grammatical if they were available. Problems with the translation software lead to too many words not having translations, but improving this was not feasible. Overall, the gloss system was effective in improving the users' understanding of the articles, and as such can be considered successful.

