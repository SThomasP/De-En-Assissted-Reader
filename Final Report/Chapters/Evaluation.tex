\chapter{Evaluation of the Project}

\section{Development}

Development of the project was mostly successful, and the first part of the project proceeded as planned. Development of the article display view and gloss were completed before the Christmas holidays, as well as the progress report detailing the state of the project at the time. Continuing as planned, the paperwork needed for ethics approval was completed over the Christmas holidays. 

Once the Christmas holidays were over, the fact that user testing had to have happened before the Easter holidays and the fact that the main part of the application had already been developed, meant that the development of the project went off the planned schedule. The article discovery system being scaled back several times to meet the time constraints. The original idea for this was to build a system that would recommend articles based on a combination of difficulty and category using machine learning algorithms. The main problem with this idea, as well all other ideas that used machine learning algorithms was primarily the fact that there would not be enough data gathered before user testing for the recommender system, or any machine learning system to work correctly. So the application was scaled back as the various grand ideas proposed were not implementable.

User testing was planned to have more than four testers and the fact that only those four could be found was disappointing. The testing period was advertised to German learners of suitable experience levels but only one of those people turned up. The other testers having been recruited through other means. 

The Survey questions as well could have been done better, they reflected an application that was expected to have a recommender system and then they were hastily edited to be more in line with the application as was shown to testers. The decision to have them only read three articles then fill out unique questions about each of those articles could have been replaced with no article limit, and then having survey questions about the application in general, aggregating the results of all three articles.

The method of connecting survey results to usage data was very badly done, having the user be given an id number by the application and then having them enter it into the survey. This could have definitely been improved as the logging data was barely used in the actual analysis, instead having the user simply entering their experience level again in the survey would have been a better option. 

\section{The Application}

The application started development incorrectly. Rather than developing it as a web application, the application should have, as one of the testers suggested, been developed as a browser extension. Having the user come across unknown  words while browsing German websites and then selected then and looking them up in the application would have most likely been a better application. This application could still have suggested a number of articles as starting points and then just redirected the user to the article on the site. 

The decision after the initial gloss was developed to build the article discovery system was also a mistake, more effort should have been put into developing advanced parsing, better translations and contextually determining both the grammatical features of the word as well as more accurate translations of the word. A more accurate, contextual gloss would have likely been a better product than the generic gloss and discovery system that was developed.

The application that was developed, while built on shaky premises, was functional and in spite of all the bugs, it fulfilled the goals of the project. It was designed decently and the gloss system worked. The article discovery system of the application was also a functional implementation that fulfilled the goals of the project. There are a lot of ways to improve it, but the way that it was developed allowed for the discovery of articles in a variety of categories from various German language news sources. These articles could then be read by the user with an attached gloss.

