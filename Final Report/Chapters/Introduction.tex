\chapter{Project Description}
\section{Problem}
A gloss is a language learning tool that assists with increasing the user's reading comprehension and vocabulary size by proving a prompt related to a selected word. The prompt can  can have a variety of contents such as a translation into the author's native language, a definition in the source or various pieces of audio-visual media.

There have been a number of gloss tools developed for assisting a learner with second language learning. Most of these tools are tailored to a text or set of texts and cannot be applied generally, this means that there is no guarantee that the learner will find the glossed texts interesting, leading to the learning not engaging with the text to the extent that they might if they enjoyed reading about the subject of the text. 

A learner, especially an intermediate learner will struggle to perform the reading required on news site home pages and aggregate sites to find texts related to their interests and of a suitable to difficulty level.

\section{Goals}

The goal of this project is to develop a program that will provide a custom gloss that can be applied to every article, only glossing text that the learner requests. The gloss will then display information related to the word that has been selected, this information can then be dismissed by the learner.

The program will find and list articles to user, based on a selected category. It will also apply a difficulty rating to those articles, rating them to their relative difficulty to the user.

\section{Scope}

The scope of this project is the building of a web application that fills the above goals. It will allow the leaner to input a category and their experience level. The application will then show a list view of articles in that category as well as the article's difficulty in relationship to the user.

Once on the article view, the leaner can then select a word by clicking on it, the application will then show a gloss with information about the selected word. Once it is no longer needed, this information can be dismissed. The learner will then be able return to the  article selection screen, to select another article and any words in the gloss at that time will preserved in the new article. 