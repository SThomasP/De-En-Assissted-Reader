\chapter{Introduction}

\section{Problem}
While learning a second language, many learners encounter problems while trying to find and read texts written in the language they are learning. Modern technology has helped with this to some regard, with development of glosses, which are digital tools that provide prompts for vocabulary words. These prompts can be translations of the words, definitions in the source language, audio visual materials or any combination of the above. 

The problem most glosses developed today have is that they are specifically tailored for the text they are being used on. The words that can be examined are pre-selected, the gloss content is pre-written and the text has been pre-selected. This still leaves multiple problems for the user: They might not find the text interesting, they may already know the words in the gloss or they might find the text too easy. 

\section{Solution and Goals}

The solution to this problem was, perhaps unsurprisingly, to develop a prototype application. This application should be able to:

\begin{itemize}
	\item Allow the user to a select a category of articles that they find interesting.
	
	\item Find and pull articles in that category from a variety of news sources.
	
	\item Rate the articles based on their perceived difficulty for the user.
	
	\item Allow the user to activate a gloss on any and all words in the article. 
\end{itemize}

To make sure as many platforms as possible are supported as easily as possible, the application will run in a web browser and be hosted remotely. The initial prototype of the application will support German alone in order to minimise the number of variables that need to be taken into account and because German is a language the developer can speak, allowing for easier development of the application.

Once developed, this  prototype application was then user tested by a number of German learners of varying experience levels. They then provided feedback on their experiences using the application. From this data, the idea of a potential release build was then formulated. 