\begin{lstlisting}[caption={[Grammatical Features Extraction Code] Python code that extracts and sorts the grammatical features given by \textit{Oxford Dictionaires API}}, label=lst:gramm, breaklines=true, language=python]
@staticmethod
def sort_grammar(gram_fe):
    counter = {}
    # count the occurrences of each grammatical type
    for feature in gram_fe:
        g_type = feature['type'].lower()
        if g_type in counter.keys():
            counter[g_type] += 1
        else:
            counter[g_type] = 1
    # determine the maximum
    if counter == {}:
        return []
    maximum = max(counter, key=(lambda key: counter[key]))
    # create the empty dictionaries
    sorted = [{} for _ in range(counter[maximum])]

    # create another dictionary to keep track of how many times a type has occurred while sorting.
    occurred = {}
    for key in counter.keys():
        occurred[key] = 0

    # sort the features
    for i in range(len(gram_fe)):
        g_type = gram_fe[i]['type'].lower()
        text = gram_fe[i]['text'].lower()
        o = occurred[g_type]
        sorted[o][g_type] = text
        if o == counter[g_type] - 1:
            for j in range(o, len(sorted)):
                sorted[j][g_type] = text
        occurred[g_type] += 1

    # some stuff that has to be coded manually
    if 'person' in sorted[0].keys() and 'number' in sorted[0].keys() and len(sorted) >= 2:
        if sorted[0]['person'] == 'second' and sorted[1]['person'] == 'third':
            sorted[0]['person'] = 'third'
            sorted[1]['person'] = 'second'

    if 'degree' in sorted[0].keys() and len(sorted) > 1:
        if sorted[0]['degree'] == 'positive' and sorted[1]['degree'] == 'comparative':
            sorted[0]['degree'] = 'comparative'
            sorted[1]['degree'] = 'positive'

    return sorted
\end{lstlisting}
