%% ----------------------------------------------------------------
%% Progress.tex
%% ---------------------------------------------------------------- 
\documentclass{ecsreport}  
\usepackage{libertine}
\usepackage{rotating}
\usepackage[ngerman,english]{babel} % Use the progress Style
\newcommand{\code}[1]{\texttt{#1}}
\usepackage[backend=biber, style=authoryear]{biblatex}  % Location of your graphics files
\addbibresource{Bibliography.bib}
\usepackage{tabu}
\usepackage{float}
\hypersetup{colorlinks=true}   % Set to false for black/white printing
%
\definecolor{purple}{rgb}{0.65, 0.12, 0.82}
\lstset{
	breakatwhitespace=true,
	breaklines=true,
	keywordstyle=\color{blue}\bfseries,
	identifierstyle=\color{black},
	commentstyle=\color{purple},
	stringstyle=\color{red},
}

\lstdefinelanguage{javascript}{
	keywords={break, case, catch, continue, debugger, default, delete, do, else, false, finally, for, function, if, in, instanceof, new, null, return, switch, this, throw, true, try, typeof, var, void, while, with},
	morecomment=[l]{//},
	morecomment=[s]{/*}{*/},
	morestring=[b]',
	morestring=[b]",
	ndkeywords={class, export, boolean, throw, implements, import, this},
	sensitive=true
}            % Include your abbreviations
%% ----------------------------------------------------------------
\begin{document}
\frontmatter
\title      {A Gloss and Difficulty Rating System to Assist with the Discovery and Reading of German Language Articles}
\authors    {\texorpdfstring
             {\href{mailto:stp1g15@ecs.soton.ac.uk}{Steffan T. Padel}}
             {Steffan T. Padel}
            }
\addresses  {\groupname\\\deptname\\\univname}
\date       {\today}
\subject    {}
\keywords   {}
\supervisor {Prof. Adam Prugel-Bennett}
\examiner   {Prof. Lie-Liang Yang}
\maketitle
\begin{abstract}
	
This report proposes and then details the development and testing process of a gloss application designed to help German language learners discover and then read authentic articles from German news sources. It begins by outlining the application before starting research into the logic and methods such an application would use. Once these have been decided on, the project then goes on to detail how the application was designed and programmed, listing the choices made and the reasoning behind them. The report then details and analyses the results of user testing on the application, before performing a critical evaluation of the both the application and the project as a whole, detailing their flaws and the various methods by which they could be improved. The report then concludes by examining how the application could be improved for a future commercial version before deciding that such improvements are not viable.	 
	
\end{abstract}
\tableofcontents
\listoffigures
\listoftables
\lstlistoflistings
\mainmatter
%% ----------------------------------------------------------------
\chapter{Project Description}
\section{Problem}

There have been a number of gloss tools developer for assisting a learner with L2 learning. Most of these tools are tailored to a text or set of texts and cannot be applied generally, this means that there is no guarantee that the learner will find the glossed texts interesting, leading to the learning not engaging with the text to the extent that they might if they enjoyed reading about the subject of the text. There may also be words not in the gloss that the learner does not know, and words in the gloss that the leaner already knows. 


A learner, especially an intermediate learner will struggle to perform the reading required on news site home pages and aggregate sites to find texts related to their interests. 
\section{Goals}

The goal of this project is to develop a program that will provide a custom gloss that can be applied to every article, only glossing text that the learner requests. The gloss will then display information related to the word that has been selected, this information can then be dismissed by the learner.

The program will also suggest articles to the learner based on the previously read articles and selected categories. 



\section{Scope}

The scope of this project is the building of a web application that fills the above goals. It will allow the leaner to select categories, and then recommend articles to them based on this selection. The user will then select an article and the article will then be presented to the user in a readable format.

Once on the article view, the leaner can then select a word by clicking on it, the application will then show a gloss with information about the selected word. Once it is no longer needed, this information can be dismissed. The learner will then be able return to the  article selection screen, to select another article and any words in the gloss at that time will preserved in the new article.          % Proofread 1
\chapter{Application Development Plan}

A visualization for the plan for this project can be seen in the Gannt chart in figure \ref{fig:gannt}
\begin{figure}
	\caption[Gannt Chart of the Project]{Gannt chart showing the project's planned process over the academic year.}
	\label{fig:gannt}
	\begin{center}
	\includegraphics[width=\textwidth]{Graphics/Gannt}
	\end{center}
\end{figure} 

\section{Readings and Research}

The plan details that reading was to be divided into two parts, reading about glosses and reading about difficulty rating systems. With the reading about difficulty rating systems being done after the gloss was developed. This was done as the difficulty rating system was not part of the minimum viable project, so the reading about it could be left until after the development of the minimum viable project was complete.

\section{Development}

The application was developed using AGILE methods. The project was divided into six sprints and they were completed in the order in which they are described. The order was chosen as it would allow the application to reach minimal viable project status as soon as possible (end of sprint 3), leaving the more advanced sprints for later. The sprints are detailed below.

\begin{enumerate}
	\item \textbf{Translation Lookup}
	
	The aim of this sprint is to establish a valid method to retrieve single word translation and grammatical data.
	
	\item \textbf{Article Parsing \& Presentation}
	
	The aim of this sprint is develop a method for the parsing of articles and then to present them through a web browser
	
	\item \textbf{AJAX}
	
	The aim of this sprint is to develop the javscript code that will allow the gloss to display entries.
	
	\item \textbf{Advance Parsing}
	
	The aim of this sprint is to allow advanced parsing methods to be performed on the text. Allowing for parts of speech identification of words. 
	
	\item \textbf{Article Discovery}
	
	The aim of this sprint is get the application to find articles on it's own, removing the need for the user to find their own articles.
	
	\item \textbf{Difficulty Rating System}
	
	The aim of this sprint it to develop the system that allows the application to rate how difficult the user will find various articles. 
	
\end{enumerate}

AGILE methodologies were used as they allowed the developer to have regular meetings with the project supervisor, discussing the progress made, what was left to be done whether or not it was achievable.

\section{Testing}

Three types of testing were done on the application. Unit testing was during the development of sprint one, to ensure that the application was interacting with the dictionary API correctly, making sure that the correct translations of various words were found. 

functional testing was during the development of the rest of the application, the developer would use the application, testing whether or not the feature they were currently implementing was working as planned.

Finally, user testing was done once all six sprints were completed, it consisted on the user using the application to read three articles, and then filling out a feed back survey about their experiences with it. In addition, the user's use of the application was tracked, seeing what words they requested and what articles they read. Ethics approval for this test was obtained from the ECS ethics board.           % Proofread 1
\chapter{Research \& Background Reading}
The body of literature surrounding electronic glosses is rich in varied. The research done for the sake of this project is structured in two parts.
\begin{enumerate}
	\item Whether or not electronic glosses are effective in developing the skills of reading comprehension and vocabulary retention.
	\item A look at which design aspects of electronic glosses are most effective in developing those skills.
	\item Background research into recommender systems (Not done at time of progress report) 
\end{enumerate}

\section{Effectiveness of Electronic Glosses}
First, it should be established whether or not an electronic gloss might be an effective for increasing vocabulary retention and reading comprehension. If not, then there is little to no reason in continuing in development of this project, as it will fail to be effective in its goals. 

\textcite{abraham2008} finds that, overall, that learners with access to an electronic gloss will perform better than learners without access, in both skills of reading comprehension and vocabulary retention, particularly on intermediate learners. They note that the small sample size of studies analysed should mean that their study is not seen as conclusive evidence, however suggestive evidence is enough to establish that development of this project should continue.

Another caveat that should also be added that most of the studies analysed in \textcite{abraham2008} are of tailor-made glosses designed for a specific text, as such the research findings of these studies and the meta-study may not be fully applicable to this project.


\section{Design Methods of Electronic Glosses}

Next it should be established the best way to design a gloss to most effectively increase the desired skill set in the learner. \textcite{roby1999} identifies three main parts of a gloss' design, its presentation, its taxonomy and its density. However, as density for this project's application is something determined by the leaner, background reading into design was limited to presentation and taxonomy. 

\subsection{Gloss Presentation}
\textcite{roby1999} defines gloss presentations is how the gloss is presented relative to the text, primarily this is about the location of the gloss content, however other factors such as colour schemes are also part of it.

\textcite{chen2016} identifies the three most researched gloss presentations as: in-text, marginal and pop up.  \textcite{abuseileek2008} finds that out of these three the marginal presentation category performs the best in both vocabulary retention and reading comprehension, while \textcite{marefat2016} finds that pop-ups are more effective than marginal for reading comprehension. As \textcite{chen2016} says, there has not been sufficient research into gloss location for valid conclusions to be drawn. 

\subsection{Gloss Taxonomy}
Gloss taxonomy is the content of the gloss. What is presented 

    % Proofread 1
\chapter{Proposed Application}

Below is a rough outline of the application and list of the development sprints for the application. More detailed descriptions of both the design of the application and the server side logic of the application are presented in later chapters. 

\section{Outline}

The outline of the application is as follows: 

The user starts the application by selecting a category of articles, based on their interests, as well as their level of experience with German. From there, they are presented with a list of articles in that category, with each article having a difficulty rating. The user can then select an article based on the title of the article as well as the difficulty rating.

Once the user has selected the article, they are presented with the text of the article. From here, they can click on words in the article, and the application will show them a translation of the word and grammatical information about the word in a marginal gloss form. These gloss element can then be removed from the screen when no longer needed.

A systems flow diagram of the basic outline of the application can be seen Figure \ref{fig:sf}

\begin{figure}
	\caption[Systems Flow Diagram of the Application]{Systems flow diagram of the application process, where the user finds and reads articles using the application and the methodology of how the application achieves this functionality.}
	\label{fig:sf}
	\begin{center}
	\includegraphics[width=0.7\textwidth]{Graphics/SystemsFlowNew}
\end{center}
\end{figure}

\section{Development}

Development of the application was divided into six sprints and they were completed in the order in which they are described. The order was chosen as it would allow the application to reach minimal viable project status as soon as possible (end of Sprint 3), leaving the more advanced sprints for later. The sprints are detailed below.

\begin{enumerate}
	\item \textbf{Translation Lookup}
	
	The aim of this sprint is to establish a valid method to retrieve single word translation and grammatical data.
	
	\item \textbf{Article Parsing \& Presentation}
	
	The aim of this sprint is develop a method for the parsing of articles and then to present them through a web browser
	
	\item \textbf{AJAX}
	
	The aim of this sprint is to develop the JavaScript code that will allow the gloss to display entries.
	
	\item \textbf{Advance Parsing}
	
	The aim of this sprint is to allow advanced parsing methods to be performed on the text. Allowing for better identification of the words and their translations. 
	
	\item \textbf{Article Discovery}
	
	The aim of this sprint is get the application to find articles on it's own, removing the need for the user to find their own articles.
	
	\item \textbf{Difficulty Rating System}
	
	The aim of this sprint it to develop the system that allows the application to rate how difficult the user will find various articles. 
	
\end{enumerate}
   % Proofread 1
\chapter{Application Design}

This chapter details with the user interface and design elements of the application, specifically how the application appeared when the user testing took place. It details the design decisions made and then proceeds to explain and justify why it was thought that making these decisions would improve the user experience.

\section{Category and Experience Selection}

When the user first connects to the web application, they are shown the view in figure \ref{fig:view1}. This view allows them to input their experience with German, as well as selecting the category of article that they wish to read. Four different experience levels are available, "beginner", "intermediate", "advanced" and "near fluent". The decision to categorize language experience this was made for a number of reasons. First and for most, the names of the different experience levels are in English and are commonly used as descriptions of language proficiency. While it would have be possible to use much more formal definitions of language proficiency, for example the Common European Framework of Reference for Language (CEFRL). These definition are not commonly taught and would most likely confuse a large proportion of users. Once "beginner", "intermediate" and "advanced" had been decided upon, the decision was made to add a forth level "near fluent" to distinguish between users  who needed barely any help and users who needed some help. It was also believed that fluent German speakers would have no need to use the application, and therefore did not need to be considered when designing these levels.

\begin{figure}
	\caption{Screenshot of the Category Selection View}
	\label{fig:view1}
	\begin{center}
	\includegraphics[width=0.7\textwidth]{Graphics/View1}
	\end{center}
\end{figure}

The user then goes on to select a category of article. The Selections available are "All", "Business", "Entertainment", "Germany", "Health", "Science and Tech", "Sports" and "World". These categories were chosen to reflect the traditional new categories that one would find on a news website. The decision was made to use English titles rather than German ones, to allow a range of users to able to understand what the categories are. German titles were used briefly during a stage of development, but were switched to English as it was believed that some users would not be skilled enough in German to understand the meaning of all the titles without access to translations, which are not currently available on this screen. 

A symbol was added next to the name for each category. Primarily to make it easier to identify the categories upon sight, but also to broaden the colour pallet being used in the application.

The initial prototype of the application had a different article selection screen, Where the URL of the desired article was inputted instead of a category selection screen. This had both advantages and disadvantages over the selection model in the final screen. This allowed the user to find articles that interested them and then input them into the application and continue reading there, but it also relied on the user having to be able to find article that interested them externally. Ideally, it would have been best to have both the category selection and direct article input methods implemented, perhaps with different input screens, however, due to time constraints. This was not implemented.

\section{Article Selection}

Once the user has inputted their ability level and desired category of article, they are then presented with the view shown in figure \ref{fig:view2}. This is a list of articles in the selected category, with titles in German, followed by a difficulty rating, also in German. Titles are left in the original German as the level of reading required to read the title is most likely less than the reading level of the user, allowing them to select their own articles. If not, the difficulty ratings (also in German) provide a clear indication of the level of the articles. 

\begin{figure}
	\caption[Screenshot of the Article Selection View]{An example of the article selection view, here the user is shown a list of articles in their selected category as well as the difficulty rating for each article. They then go on to select an article from the list.}
	\label{fig:view2}
	\begin{center}
	\includegraphics[width=0.7\textwidth]{Graphics/View2}
\end{center}
\end{figure}

To right of the list of articles, in the position where the gloss will be in the article view, a list of definitions of the different difficulty rating can be seen. These, going in order from most difficult to least are "SS", "S", "ZS", "M", "ZL", "L" and "SL". These difficulty ratings were chosen as they are acronyms of German phrases that translate to the appropriate difficult levels, each one of them having a unique initialisation. Seven categories were used, expanding the standard "Easy", "Medium" and "Hard" to include "Very Hard", "Somewhat Hard", "Somewhat Easy" and "Very Easy". These categories can be placed easily on a scale by a user, and then be used to identify with a fair amount  of precision how hard said end user will find the article.

The decision to include the definition box was to make sure that the user could map the German initialisation to the English category definition as no full German words are used in the rating labels and the end user might not be of level where they can perform this mapping without assistance.  

\section{Article Reading}

Once the user has selected an article from the list, they are shown are the view in figure \ref{fig:view3} which is primarily the content of the article. A button to take the user back to list view is shown to left of the article. On the right of the article text, there is a box which prompts the user to click on words. An additional visual prompt is provided, the cursor changes to the common "pointer" cursor when over a word in the article, suggesting that the word can be clicked on.


\begin{figure}
	\caption{Screenshot of the Article Reading View}
	\label{fig:view3}
	\begin{center}
	\includegraphics[width=0.7\textwidth]{Graphics/View3}
\end{center}
\end{figure}

Clicking on a word on the article, will make a gloss item to appear on the right of the screen, making the screen appear similar to the view shown in figure \ref{fig:view4}. This position is a marginal gloss, which, is a form a gloss that is effective \autocite{abuseileek2008}. New items appear at the bottom of the column, permanently, until dismissed. If there are too many gloss entries, then a scroll bar will appear, allowing the user to scroll down. These gloss entries are permanent until dismissed to prevent the user from having to lookup the same word multiple times.

\begin{figure}
	\caption[Screenshot of the Article Reading View with Gloss]{Another screenshot of the article reading view, this time with a gloss entry in the margin.}
	\label{fig:view4}
	\begin{center}
	\includegraphics[width=0.7\textwidth]{Graphics/View4}
\end{center}
\end{figure}
 

\subsection{Gloss Items}

\begin{figure}
	\caption{Screenshot of a Gloss Entry}
	\label{fig:gloss}
	\begin{center}
	\includegraphics[width=0.3\textwidth]{Graphics/Gloss}
\end{center}
\end{figure}

The design of the gloss items was inspired by the look of the standard dictionary entry, an example of the design can be seen in figure \ref{fig:gloss}. The top of the gloss is a colour specified by the lexical category of the word; green for nouns, purple for verbs, blue for adjectives, red for any others and black for ones that cannot be identified. The header then contains the form of the words as it appears in the text, followed by the lexical category, in curly brackets, with unique text and background colours. A white cross is on the right of the header, clicking it dismisses the gloss entry, removing it from the gloss. 

The body of the gloss is divided into three parts. The root word, the grammatical information and possible translations of the root. The root is presented at the top, in bold. This is done so that if the user can identify the word by the root, than they only need to read that. 

After the root, the grammatical information of the word. First is a short description of the word's lexical category and form of the word. Following this is a bullet point list of the possible reasons the word has mutated from the root; Its tense and person if it's a verb, gender, plurality, and case if it's a noun and other information for other lexical categories. 

Finally the body is concluded with various translations of the root words \autocite{gettys2001}, including all senses of the word. In the case that the program cannot find a translation, the phrase "no translations found" is presented instead. 
     %
%TC:envir lstlisting [] xall
\chapter{Application Implementation}

\section{General Technologies and Frameworks}

As it is a web application, HTML, CSS and JavaScript were used for the client side of the application, although this was augmented with both \textit{JQuery} and \textit{Bootstrap} to ease the development of the Javascript and the CSS respectively. 

For the server side of the application, the code was written in Python, using the \textit{Flask} framework and its integrated technologies. \textit{Flask} was used as it is easier to use then the alternative Python frameworks and the developer has experience using it. Python was also chosen for similar reasons, including the developer's experience with it. The memory usage and speed of the various possible programming languages and frameworks was not considered. Ease of use and familiarity were the primary concerns. 

\section{Article Discovery}

For the article discovery portion of the application, the need was for a list of articles, from a variety of sources about a variety of topics. The solution that was found after some searching on the internet, was \textit{Google News}.

\textit{Google News} provides several lists of recent articles from different sources divided in different topics, additionally, each of these lists is provided in an RSS feed XML file, a format that allows the code to be written for both URL and title extraction with minimal effort.

The categories provided by the \textit{Google News} RSS feeds were usable, but overall they were fairly common news categories. On the plus side, these categories are recognisable, on the down side, these categories are not customisable, however, the categories were pre-selected, saving time on artificial category creation, the end user can just selected one of them. The code for obtaining the articles is shown in Listing \ref{lst:disc}.

\begin{lstlisting}[caption=Article Discovery Code, label=lst:disc, breaklines=true, language=python]
    def lookup(self, category, user_level):
        r = requests.get(CAT_MAP[category]['url'])
        root = ElementTree.fromstring(r.text)
        old_aritcles = [a.id for a in self.articles[category]]
        for item in root.iter('item'):
            url = item.find('link').text
            title = item.find('title').text
            aid = str(uuid.uuid5(uuid.NAMESPACE_URL, url))
            if aid not in old_aritcles:
                try:
                    self.articles[category].append(Article(url, title, aid, category))
                except ZeroDivisionError:
                    print('Zero Divison Error')
                except ArticleException:
                    print('Article Exception')
        return self.classify(self.articles[category], user_level)
\end{lstlisting}


This code takes the category, and then gets the current state of the corresponding RSS feed. It then iterates through item tags in the RSS feed, extracting the title and URL of each tag. It then attempts to parse and rate the article for listing. Once every article has be parsed and rated, it returns a list for display. In addition, articles are stored and added to future lists so that the code doesn't have to reparse an article every time the list is generated.

There were alternatives to the \textit{Google News} RSS feeds that were examined, primarily \textit{NewsAPI}. However this was found to be lacking for two reasons. Primarily, this API would list content, where the primary means of communication was not text. Comics and other image-based media got added to the lists being provided by the API. As the parser and the gloss system relied on the content being text based and there was no easy way to distinguish between text based content and image based content. The other more minor reason that \textit{NewsAPI} was disregarded was because it was more difficult to categorise the articles, whereas the \textit{Google News} RSS feeds provided pre-created categories, to save time on both filtering out image-based content and the categorisation of articles, the decision was made to use the \textit{Google News} RSS feeds. 


\section{Article Difficulty Rating}

The difficulty rating of the article is calculated using two things, the article text as well as the user's experience level. The user's experience level is obtained from the form the user submitted at the start, while the article text needs to be extracted from the article web page and then analysed.

To extract and then analyse the article, two technologies are used. The first is an article scraping Python library called \textit{Newspaper}, the purpose of it is to extract the raw article text from the article web page, making the analysis of it easier. Analysis of the article text is then done through another P ython library called \textit{SpaCy}, which is a parts-of-speech tagging library, while an add-on for it, called \textit{Textacy} provides advanced features such as syllable counting. \textit{SpaCy} provides two  sets of parts of speech tags with different levels of precision, which are referred to as fine-grained and coarse-grained.

Once the article text has been extracted and analysed, it's readability index needs to be calculated, this is done using the Erste Wiener Sachetextformel (WSTF) as described in \textcite{bamberger1984}. The WSTF formula was chosen over the FRE formula because the WSTF formula has a smaller output range. This smaller output range allowed for two things the first was easier mapping onto the user experience level as well as a better understanding of the approximate level of each individual number, as those number correspond directly with the German school years. 

Once the readability score is calculated, it needs to be mapped, using the user experience level onto the difficulty ratings, a simple linear map was chosen for this, the mappings can be seen in Figure \ref{fig:ratings}.

\begin{figure}
	\caption[Aritcle Rating Mappings]{Mappings showing the article readability score against experience level and the ratings decided upon based on those variables.}
	\label{fig:ratings}
	\begin{center}
	\includegraphics[width=0.7\textwidth]{Graphics/Ratings}
\end{center}
\end{figure}

The primary idea behind these ratings was that a user who was less experienced with German would find have a lower threshold for articles that they find difficult. To do this it was decided that near fluent user would be able to read with some confidence articles with a WSTF score of 10 which corresponds to a native speaker who is 15/16 years old. The rest of the mapping were calculated then using this as baseline. 

More complex ideas for mapping were experimented with, including ones that used machine learning systems, however these ideas were eventually abandoned due to the fact that they were to complex to be implemented in the limited time available. 

\section{Displaying the Article}

Once the users has selected an article, they are then presented with the content of that article. The actual job of presenting this article was fairly easy as the was majority of this work (the extraction and parsing of the article's content) had already been done by \textit{newspaper} and \textit{SpaCy}.  \textit{SpaCy's} method of presenting the parsed article is as a series of tags, each representing a part of speech. This makes presenting the article easy, the tags are iterated through as if they were a list. Words are wrapped in span elements tags for later use, punctuation is skipped over and paragraph breaks end the current paragraph and start a new one. In addition these span tags have data attributes storing the parts of speech tags and root word. This results in a html document that appears similar to Listing \ref{lst:html}.

\small{\begin{lstlisting}[language=HTML, caption={[Example HTML Code] An example of how the HTML code of the article ended up appearing after it had been formatted.}, label=lst:html, breaklines=true, breakatwhitespace=true, float]
<p>
	<span class="word" data-lemma="Der" data-tag="ART" data-pos="DET">Der</span>
	<span class="word" data-lemma="Index" data-tag="NN" data-pos="NOUN">Index</span>
	<span class="word" data-lemma="der" data-tag="ART" data-pos="DET">der</span>
	<span class="word" data-lemma="30" data-tag="CARD" data-pos="NUM">30</span>
	<span class="word" data-lemma="groß" data-tag="ADJA" data-pos="ADJ">größten</span>
	<span class="word" data-lemma="Aktiengesellschaften" data-tag="NN" data-pos="NOUN">Aktiengesellschaften</span>
	<span class="word" data-lemma="steigen" data-tag="VVFIN" data-pos="VERB">stieg</span>
	<span class="word" data-lemma="heute" data-tag="ADV" data-pos="ADV">heute</span>
	<span class="word" data-lemma="zeitweise" data-tag="ADV" data-pos="ADV">zeitweise</span>
	<span class="word" data-lemma="bis" data-tag="APPR" data-pos="ADP">bis</span>
	<span class="word" data-lemma="auf" data-tag="ADV" data-pos="ADV">auf</span>
	<span class="word" data-lemma="12.524,97" data-tag="CARD" data-pos="NUM">12.524,97</span>
	<span class="word" data-lemma="punkten" data-tag="NN" data-pos="NOUN">Punkte</span>
	,
	<span class="word" data-lemma="entsprechen" data-tag="ADJD" data-pos="ADJ">entsprechend</span>
	<span class="word" data-lemma="einer" data-tag="ART" data-pos="DET">einem</span>
	<span class="word" data-lemma="gewinnen" data-tag="NN" data-pos="NOUN">Gewinn</span>
	<span class="word" data-lemma="von" data-tag="APPR" data-pos="ADP">von</span>
	<span class="word" data-lemma="0,13" data-tag="CARD" data-pos="NUM">0,13</span>
	<span class="word" data-lemma="Prozent" data-tag="NN" data-pos="NOUN">Prozent</span>
	.
</p>

\end{lstlisting}}

The html produced by this was functional, although there was a bug to do with the generation of excess whitespace, which was removed from Listing \ref{lst:html} so that the code was readable. 

\section{Gloss Creation}

The client side method for creating the gloss is done through AJAX, the javascript for this can be seen in Listing \ref{lst:gloss}.

\begin{lstlisting}[caption=Gloss Javascript, label=lst:gloss, breaklines=true]
$('.word').on('click', function (event) {
	var e = event.target;
	var data = e.dataset;
	data.word = e.textContent;
	$.ajax({
		url: "/dict",
		type: 'POST',
		data: data,
		success: function (result) {
			popupEntry(result);
		}
	});
});

function popupEntry(entry) {
	var entryList = document.getElementById("dict-entries");
	entryList.insertAdjacentHTML('beforeend', entry);
}

\end{lstlisting}

This code is simple, when one of the words is click, an AJAX event is fired. It requests the gloss item from the application, and upon return inserts the gloss item at the end of the marginal gloss.

Upon receiving such a gloss request, the server side python code will then generate and return a pre-formatted gloss. This allowed for the client side code to be kept as simple as possible. The code for this is shown in Listing \ref{lst:dict}

\begin{lstlisting}[caption={[Gloss Generation Code] Python code used to generate the gloss content and organise so it can be formatted into a gloss item.}, label=lst:dict, breaklines=true, language=python]
class DictEntry:
    def __init__(self, word, lemma, tag):
        self.word = word
        self.root = lemma
        self.tag = tag
        self.pos = TAG_DICT[tag]
        if self.pos in ['Noun', 'Verb', 'Adjective', 'Unknown']:
            self.css_cat = self.pos.lower()
        else:
            self.css_cat = 'other'

        if self.pos not in ['Proper Noun', 'Other', 'Numeral']:
            self.found, translation, grammar = dictionary.lookup(word, lemma, self.pos)
            if self.found:
                self.english = self.gen_english_string(translation)
                self.grammar_features = self.list_features(grammar)
            else:
                self.english = 'No translation found'
                self.grammar_features = []

        else:
            self.found = False
            self.english = 'Not translatable'
            self.grammar_features = []
        self.grammar_explanation = spacy.explain(tag)
\end{lstlisting}


The code starts by assigning the word, root and fine-grained part of speech tag to the class, the fine-grained part of speech tag is then used to lookup the coarse-grained one, which is then saved in a format which is both human readable and how the \textit{Oxford Dictionaries API} classifies its lexical categories, which are functionally renames of the coarse-grained part of speech tags.  If the lexical category is one where a translation will not be found (Proper Nouns, Numerals, etc.). It does not bother performing either lookup. It then calls the dictionary object (defined outside of scope) and looks up the word and grammatical information. When this information is returned, it then proceeds to format it so it is human readable.

The object where all this has be stored is then passed to the template engine. Where it is transformed into a HTML gloss item. This is then sent to the user as a response to the initial request, where the JavaScript in Listing \ref{lst:gloss} will insert it into the gloss. 

\subsection{Translation Lookup}

For the translation section of the gloss creation, various possibilities were considered, These included online translation services such as \textit{Google Cloud Translate} and \textit{Microsoft Translator}, raw datasets such as the \textit{DictCC dataset} and Dictionary APIs such as the \textit{Oxford Dictionaries API} and the \textit{Collins Dictionary API}. 

The technology that would be used in the project would have to:
\begin{itemize}
\item Be able to translate single words from German to English.
\item Provide lexical information of those words.
\item Be allowed for the content to be hosted and provided through a web interface.
\item Be available for less than \pounds150 total.
\end{itemize}

The five technologies mentioned above were checked against these criteria and the results are show in Table \ref{tbl:comp}

\begin{table}[H]
\centering
\caption[Comparison of Translation Software]{Comparison of various translation solutions to see whether or not they fulfil the criteria of the application. }
\label{tbl:comp}
\begin{tabu} to \textwidth{|X[c]|X[c]|X[c]|X[c]|X[c]|}
\hline
\textbf{Product}        & \textbf{German to English Translations} & \textbf{Lexical Information} & \textbf{Allowed Online} & \textbf{Less Than \pounds150  (total)} \\ \hline
Google Cloud Translate  & Yes                                     & No                           & Yes                     & No                               \\ \hline
Microsoft Translator    & Yes                                     & No                           & Yes                     & No                               \\ \hline
Dict.cc Dataset         & Yes                                     & Yes                          & No                      & Yes                              \\ \hline
Oxford Dictionaries API & Yes                                     & Yes                          & Yes                     & Yes                              \\ \hline
Collins Dictionary API  & Yes                                     & Yes                          & Yes                     & No                               \\ \hline
\end{tabu}
\end{table}

As the \textit{Oxford Dictionaries API} was the only technology to clear all four criteria, the decision was made to used it for development of the application, however other technologies were used in testing the resulting code.

\textit{Oxford Dictionaries API} is a REST API where two calls are required to get the desired information. The first, gets the translations of the root words, at the same time this is used to check if a word is the dictionary, the second call, which is made if a translation is found is to get the reasons why that particular mutation from the root occurred.  The process is illustrated in the systems flow diagram in Figure \ref{fig:odsf}

\begin{figure}
	\caption{Systems Flow Diagram of the Oxford Dictionaries API}
	\label{fig:odsf}
	\begin{center}
	\includegraphics[width=0.7\textwidth]{Graphics/SystemsFlowOxford}
\end{center}
\end{figure}

On the first call, the root word, as provided by \textit{SpaCy}, gets its translations looked up. If one or more translation is discovered, the second call gets made, extracting the grammatical information which lists possible reasons why the root was transformed into its current form.

Several times throughout the grammatical feature generation and translation extraction, the results of \textit{Oxford Dictionaries API} had to be parsed in interesting ways. First of all was extracting the English translations of the words. When translations of a word are looked up using the API. It presents the results in various nested lists, first of all entries, which are all words with the same lexical category and spelling but with different meanings, then the senses or the various contexts that particular word can be used, then translation which are the valid translations of that particular word in that particular sense. As each of these are a valid translation for the word and the application has no method for distinguishing the context for that word, they should all be shown. In addition, there are exception where a word will not have any senses or a sense will not have any translations, so these have to be ignored. The code for this is in Listing \ref{lst:eng}. 

\begin{lstlisting}[caption={[Translation Extraction Code] Python code for obtaining the translations of a word from the information provided by the \textit{Oxford Dictionaries API}}, label=lst:eng, breaklines=true, language=python]
@staticmethod
def sort_english(entries):
    en = []
    for entry in entries:
	    if 'senses' in entry.keys():
		    senses = entry['senses']
		    for sense in senses:
			    if 'translations' in sense.keys():
				    translations = sense['translations']
				    for translation in translations:
					    en.append(translation['text'])
    return en

\end{lstlisting}


More complex was how the API presented grammatical features. All grammatical features from all possible use cases are truncated into a list, so these need to be reassembled into the use cases, an attempt was made to do this and the code for this can be seen in Listing \ref{lst:gramm}.

\begin{lstlisting}[caption={[Grammatical Features Extraction Code] Python code that extracts and sorts the grammatical features given by \textit{Oxford Dictionaires API}}, label=lst:gramm, breaklines=true, language=python]
@staticmethod
def sort_grammar(gram_fe):
    counter = {}
    # count the occurrences of each grammatical type
    for feature in gram_fe:
        g_type = feature['type'].lower()
        if g_type in counter.keys():
            counter[g_type] += 1
        else:
            counter[g_type] = 1
    # determine the maximum
    if counter == {}:
        return []
    maximum = max(counter, key=(lambda key: counter[key]))
    # create the empty dictionaries
    sorted = [{} for _ in range(counter[maximum])]

    # create another dictionary to keep track of how many times a type has occurred while sorting.
    occurred = {}
    for key in counter.keys():
        occurred[key] = 0

    # sort the features
    for i in range(len(gram_fe)):
        g_type = gram_fe[i]['type'].lower()
        text = gram_fe[i]['text'].lower()
        o = occurred[g_type]
        sorted[o][g_type] = text
        if o == counter[g_type] - 1:
            for j in range(o, len(sorted)):
                sorted[j][g_type] = text
        occurred[g_type] += 1

    # some stuff that has to be coded manually
    if 'person' in sorted[0].keys() and 'number' in sorted[0].keys() and len(sorted) >= 2:
        if sorted[0]['person'] == 'second' and sorted[1]['person'] == 'third':
            sorted[0]['person'] = 'third'
            sorted[1]['person'] = 'second'

    if 'degree' in sorted[0].keys() and len(sorted) > 1:
        if sorted[0]['degree'] == 'positive' and sorted[1]['degree'] == 'comparative':
            sorted[0]['degree'] = 'comparative'
            sorted[1]['degree'] = 'positive'

    return sorted
\end{lstlisting}


The code starts by counting the occurrences of each type of grammatical feature, using the maximum of those counts to calculate the number of use cases there will be. Once it's done that, it iterates through the various features assigning the nth occurrence of each type to the nth use case. If there are more use cases than occurrences of that type of grammatical feature, the last occurrence is used for the remaining use cases. Finally, there are two sets of use cases that were discovered to be wrong during development and therefore have code written to fix them. 

With the translations and grammatical information extracted, the information can then be given to the gloss generation code from Listing \ref{lst:dict}. 
     % Proofread 1
\chapter{User Testing and Feedback}
\

Once the application had been developed, it was then user tested, the application was used by various German learners and feedback on the application was then given. This chapter documents how the application was tested and the proceeds to analyse the feedback from the testing examine the successes and failures of the application.

\section{Testing Methodology}

In order to carry out this testing, University of Southampton policy required that the testing had ethics approval from the ECS ethics committee, it was applied for and approved through the ergo system, with a submission ID of 40080, being classed as a category C research project. While getting this approval took longer than expected, it was approved in good time.

The plan for testing was that the tester would test the application by using it to find and read three articles, they would then go on to fill out a short feedback form on their experiences with it. The survey asked the tester the following things. 
\begin{itemize}
	\item Whether or not the application helped them understand the articles.
	
	\item Whether or not the application helped them with words in they article that they did not know.
	
	\item Whether or not the articles were of a suitable difficulty level.
	
	\item Whether or not the translations were accurate.
	
	\item Whether or not the application was easy to use.
	
	\item Whether or not the tester liked the position of the gloss.
	
	\item Whether or not the tester would the application if available. 
	
	\item A comment box for any additional thoughts the tester may have. 
\end{itemize}


In addition the testers use of the application was logged, so that their survey results could then be compared to various information they inputted when testing the application, allowing the building of a complete picture of the tester's experience with the application. 

\section{User Feedback and Analysis}

There were four people who tested the application, of the four, two rated themselves beginner, one intermediate and one advanced. This was a decent range of experience levels, although more intermediate and advanced level testers would have been appreciated.

\subsection{Overall Feedback for the Application}

Feedback for the application as a whole was mixed, with half the testers saying that they would not use the application if it were available. One tester commented that they would prefer to the use the application as a browser extension, looking up definitions for words that they discovered while browsing the web.

There was disagreement in opinion on whether or not the application was easy to use. Half of the testers thought that it was, while another was neutral on the subject and the final one disagreed with the statement, presumably finding it difficult to use. This could presumably be solved by looking more into the design of the application as a whole, rather than focusing on the gloss.


\subsection{The Article Discover and Difficulty Rating System}

The advanced and intermediate level tester said that the articles they read were of an appropriate level for them. With the advanced level tester noting that they found the articles with higher difficulty ratings to be of a higher difficulty. 

A problem that occurred with one of the beginner level tester, where the articles listed were all but one categorised as "hard" or "very hard." This lead to a problem with that tester not being able find readable articles for them. The problem could potentially be solved by pulling from different news sources based on the users experience level. For example, a beginner level user like the one who raised the complaint would get articles from sources that specifically target children or learners. While an advanced level user would get articles from a general site.

\subsection{The Gloss and Article Reader}

There was mostly positive results on whether or not the gloss helped increase the tester's understanding of the articles. Mostly they thought that it had, although for some articles they thought that it had not. There is not enough data when comparing testers and the articles they read to reach a conclusion as to why the gloss was not helpful in those cases. It would require multiple testers reading the same article.

There was very mixed feedback as to whether or not the application helped with understanding of unknown words. The correlation in this appears to come down to the testers experience level, with the advanced tester reporting that it did not help at all, while the beginners and the intermediate reporting that it helped. This is probably to with the problems faced in terms of gloss content, which are discussed in detail below.

All the testers liked the position of the annotations, meaning that a marginal gloss was probably the best idea in terms of gloss position. However one of them notes that it would be better if the annotations scrolled automatically to the new one when clicked. Expanding on this idea. It might be possible to use this as method to dismiss the gloss items automatically as well. 

\subsection{Gloss Content}

The most common problem found with the gloss section of the application was the fact that several words were missing from the dictionary. The words missing ranged in scope, from longer compound words that are not commonly found in dictionaries, to more common, smaller words that really should be in the dictionary. However, as this was a third party dictionary api, adding these more common words to the dictionary is not really possible, the solution should probably be to find a better dictionary api. However,  most of the translations that were provided were considered by the testers to be accurate. 

Returning to the inability for compound words to be translated, there was brief talk during meetings with the supervisor meetings about using machine learning algorithms to break them into component parts and translate those, adding in this logic was dropped due to time constraints. However, due to the fact that the advanced and intermediate testers both attempted to gloss these words more often than the shorter words that were in the dictionary, developing it would have most likely been beneficial for the application overall. 

In addition, a tester noted several times that the content of the grammatical information provided in the gloss was wrong. The intent behind the design of the gloss was to provide every possible reason why a word might currently be in the form that is was in, however the tester found this confusing. It would be better going forward to try and develop a way to determine from context the exact reason the word is currently in that form. 

Overall feedback relating to this was that the dictionary content was lacking, future builds of the project should probably move on to a better system obtaining of translations and grammatical information, as the service been used in the current application lead overall to more incorrect grammatical information, missing translations and other mistakes than successful gloss attempts. 

A few problems relating to the design of the gloss were also found in the application, Primarily, the fact that long would make the cross symbol used to dismiss a gloss item disappear off the side of the gloss, making it impossible to dismiss the gloss item without zooming out. Another problem that was noted by one of the testers, was that the green colour of the noun glosses and the red colour of the other gloss implied success and failure respectively, none of the other testers noted this but the colour scheme of the gloss is easily adjustable.


          %
\chapter{Evaluation of the Project}

\section{Development}

Development of the project was mostly successful, and the first part of the project proceeded as planned. Development of the article display view and gloss were completed before the Christmas holidays, as well as the progress report detailing the state of the project at the time. Continuing as planned, the paperwork needed for ethics approval was completed over the Christmas holidays. 

Once the Christmas holidays were over, the fact that user testing had to have happened before the Easter holidays and the fact that the main part of the application had already been developed, meant that the development of the project went off the planned schedule. The article discovery system being scaled back several times to meet the time constraints. The original idea for this was to build a system that would recommend articles based on a combination of difficulty and category using machine learning algorithms. The main problem with this idea, as well all other ideas that used machine learning algorithms was primarily the fact that there would not be enough data gathered before user testing for the recommender system, or any machine learning system to work correctly. So the application was scaled back as the various grand ideas proposed were not implementable.

User testing was planned to have more than four testers and the fact that only those four could be found was disappointing. The testing period was advertised to German learners of suitable experience levels but only one of those people turned up. The other testers having been recruited through other means. 

The Survey questions as well could have been done better, they reflected an application that was expected to have a recommender system and then they were hastily edited to be more in line with the application as was shown to testers. The decision to have them only read three articles then fill out unique questions about each of those articles could have been replaced with no article limit, and then having survey questions about the application in general, aggregating the results of all three articles.

The method of connecting survey results to usage data was very badly done, having the user be given an id number by the application and then having them enter it into the survey. This could have definitely been improved as the logging data was barely used in the actual analysis, instead having the user simply entering their experience level again in the survey would have been a better option. 

\section{The Application}

The application started development incorrectly. Rather than developing it as a web application, the application should have, as one of the testers suggested, been developed as a browser extension. Having the user come across unknown  words while browsing German websites and then selected then and looking them up in the application would have most likely been a better application. This application could still have suggested a number of articles as starting points and then just redirected the user to the article on the site. 

The decision after the initial gloss was developed to build the article discovery system was also a mistake, more effort should have been put into developing advanced parsing, better translations and contextually determining both the grammatical features of the word as well as more accurate translations of the word. A more accurate, contextual gloss would have likely been a better product than the generic gloss and discovery system that was developed.

The application that was developed, while built on shaky premises, was functional and in spite of all the bugs, it fulfilled the goals of the project. It was designed decently and the gloss system worked. The article discovery system of the application was also a functional implementation that fulfilled the goals of the project. There are a lot of ways to improve it, but the way that it was developed allowed for the discovery of articles in a variety of categories from various German language news sources. These articles could then be read by the user with an attached gloss.

           %
\chapter{Conclusion and Possible Future Work}

This project set out to develop an application to allow German learners to more easily discover and read German language articles, the application developed in the process achieved this, as most of the users who tested it thought that it helped them. However, the bugs in application left it as an sub-par experience, with half the testers not interested in a commercial application. 

The application was designed and coded in a logical, structured way. although more advanced techniques that used machine learning algorithms were not used, they were considered during development cycle but were not implemented due to time constraints.

Developing the application further, in the hopes of launching commercially, would probably require a restructuring  of the project into something similar to the idea that was expressed in the evaluation. The design of the gloss entries as well as application as a whole would most likely need professional input as the current design is something that was put together using various elements from pre existing css libraries and not really something for a professional application.

The translation software of the application should be replaced as most of the bugs in the application were caused by that, so a large amount of source code would have to be rewritten to interface the application with the new translation software.

There are various legal problems with that a commercial version of the application would have to deal with, particularly the re use of the article content. A commercial application would have to make sure that the articles are licensed from their various sources as the articles, being recent, are all still in copyright.

Performing all of these actions for a commercial launch of the application would be a massive undertaking and not one worth doing at the time of writing due to financial constraints. Even if it were, the lack of intent from the testers to use the application suggests a commercial launch would likely require even more work. At some point in the future, a commercial release may be a viable option, but at that point rewriting the entire code base will probably be more feasible due to potential new technologies.            %
\appendix
\chapter{Technologies Used}

\begin{description}
	\item[SpaCy] is a natural language processing python library with a publicly available German language model. 
	
	It can be found at \url{https://spacy.io/}
	
	\item[Textacy] builds on Spacy to provide higher-level natural language processing. 
	
	It can be found at \url{http://textacy.readthedocs.io/en/stable/}
	
	\item[newspaper] is a python library that can extract the plain text content from online articles. 
	
	It can be found at \url{http://newspaper.readthedocs.io/en/latest/}
	
	\item[Flask] is microframework for hosting web applications in python. 
	
	It can be found at \url{http://flask.pocoo.org/}
	
	\item[Bootstrap] is a CSS toolkit that provides a comprehensive grid system as well as some pre-built elements. 
	
	It can be found at \url{https://getbootstrap.com/}
	
	\item[JQuery] is a JavaScript library designed to make page traversal and manipulation easier.
	
	It can be found at \url{https://jquery.com/}
	
	\item[Oxford Dictionaires API] is an API designed to provide definitions, translations and conjugations for several languages. 
	
	It can be found at \url{https://developer.oxforddictionaries.com/}
	
	\item[Google News] provides articles about recent topics from a variety of sources. 
	
	It can be found at \url{https://news.google.com/news/?ned=de\&gl=DE\&hl=de}
\end{description}
\chapter{Contents of Design Archive}
\begin{description}
	\item[articles.py] Classes for the articles and the article discovery and rating system system.
	
	\item[categories.json] Category listings, in JSON format to be easily editable.
	
	\item[dictionary.py] Dictionary lookup class and a class for storing dictionary entries.
	
	\item[gloss\_app.py] Central controller of the application.
	
	\item[Pipfile] Python dependencies, install with pipenv.
	
	\item[Pipfile.lock] Python dependencies, install with pipenv.
	
	\item[static/*] Directory containing static resources, such as JavaScript and CSS files. 
		\begin{description}
			\item[dictionary.css] CSS classes for the gloss system specifically.
			
			\item[get\_translations.js] AJAX code for getting the gloss from the application server and then displaying it.
			
			\item[style.css] CSS classes for the application as a whole.
		\end{description}
	
	\item[templates/*] Display templates for the application.
		\begin{description}
			\item[article.html] Template for the article display view.
			
			\item[entry.html] Template for individual gloss entries.
			
			\item[finish.html] Template for the finished testing view.
			
			\item[home.html] Template for the category selection view.
			
			\item[main.html] Parent template for the others, except for entry.html.
			
			\item[search.html] Template for the article listing view. 
		\end{description}
	

\end{description}
\chapter{Survey Questions}

\begin{enumerate}
	\item Please Enter Your User ID
	
	\item The application helped me understand the article.
		\begin{itemize}
			\item Article 1
				\begin{itemize}
					\item Strongly Agree
					\item Agree
					\item Neutral
					\item Disagree
					\item Strongly Disagree
				\end{itemize}
				
			\item Article 2
				\begin{itemize}
					\item Strongly Agree
					\item Agree
					\item Neutral
					\item Disagree
					\item Strongly Disagree
				\end{itemize}
			
			\item Article 3
				\begin{itemize}
					\item Strongly Agree
					\item Agree
					\item Neutral
					\item Disagree
					\item Strongly Disagree
				\end{itemize}
		\end{itemize}
	
	\item The application helped me with words in the article that I did not know.
		\begin{itemize}
			\item Article 1
			\begin{itemize}
				\item Strongly Agree
				\item Agree
				\item Neutral
				\item Disagree
				\item Strongly Disagree
			\end{itemize}
			
			\item Article 2
			\begin{itemize}
				\item Strongly Agree
				\item Agree
				\item Neutral
				\item Disagree
				\item Strongly Disagree
			\end{itemize}
			
			\item Article 3
			\begin{itemize}
				\item Strongly Agree
				\item Agree
				\item Neutral
				\item Disagree
				\item Strongly Disagree
			\end{itemize}
		\end{itemize}
		
	\item The articles recommended were of a suitable difficulty level for me. 
		\begin{itemize}
			\item Article 1
			\begin{itemize}
				\item Strongly Agree
				\item Agree
				\item Neutral
				\item Disagree
				\item Strongly Disagree
			\end{itemize}
			
			\item Article 2
			\begin{itemize}
				\item Strongly Agree
				\item Agree
				\item Neutral
				\item Disagree
				\item Strongly Disagree
			\end{itemize}
			
			\item Article 3
			\begin{itemize}
				\item Strongly Agree
				\item Agree
				\item Neutral
				\item Disagree
				\item Strongly Disagree
			\end{itemize}
		\end{itemize}

	\item I thought the translations were accurate.
		\begin{itemize}
			\item Article 1
			\begin{itemize}
				\item Strongly Agree
				\item Agree
				\item Neutral
				\item Disagree
				\item Strongly Disagree
			\end{itemize}
			
			\item Article 2
			\begin{itemize}
				\item Strongly Agree
				\item Agree
				\item Neutral
				\item Disagree
				\item Strongly Disagree
			\end{itemize}
			
			\item Article 3
			\begin{itemize}
				\item Strongly Agree
				\item Agree
				\item Neutral
				\item Disagree
				\item Strongly Disagree
			\end{itemize}
		\end{itemize}

	\item I found the application easy to use.
		\begin{itemize}
			\item Strongly Agree
			\item Agree
			\item Neutral
			\item Disagree
			\item Strongly Disagree
		\end{itemize}
		
	\item I liked the position of the annotations.
		\begin{itemize}
			\item Strongly Agree
			\item Agree
			\item Neutral
			\item Disagree
			\item Strongly Disagree
		\end{itemize}
		
	\item I would use the application if it were available.
		\begin{itemize}
			\item Strongly Agree
			\item Agree
			\item Neutral
			\item Disagree
			\item Strongly Disagree
		\end{itemize}
	
	\item Any additional thoughts.
\end{enumerate}
\include{Appendicies/OriginalBrief}

\backmatter
%\bibliographystyle{ecs}
%\bibliography{Bibliography}
\printbibliography
\end{document}
%% ----------------------------------------------------------------
