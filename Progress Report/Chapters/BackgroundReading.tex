\chapter{Research \& Background Reading}
The research done for this project is structured in three parts.
\begin{enumerate}
	\item Whether or not electronic glosses are effective in developing the skills of reading comprehension and vocabulary retention.
	\item A look at which design aspects of electronic glosses are most effective in developing those skills.
	\item Background research into recommender systems (Not done at time of progress report) 
\end{enumerate}

\section{Effectiveness of Electronic Glosses}

\textcite{abraham2008} finds that, overall, that learners with access to an electronic gloss will perform better than learners without access, in both skills of reading comprehension and vocabulary retention, particularly on intermediate learners. They note that the small sample size of studies analysed should mean that their study is not seen as conclusive evidence, however suggestive evidence is enough to establish that development of this project should continue.

Another caveat that should also be added that most of the studies analysed in \textcite{abraham2008} are of tailor-made glosses designed for a specific text, as such the research findings of these studies and the meta-study may not be fully applicable to this project.


\section{Design Methods of Electronic Glosses}

 \textcite{roby1999} identifies three main parts of a gloss' design, its presentation, its taxonomy and its density. However, as density for this project's application is something determined by the leaner, background reading into design was limited to presentation and taxonomy. 

\subsection{Gloss Presentation}
Gloss presentations is how the gloss is presented relative to the text. 

\textcite{chen2016} identifies the three most researched gloss presentations as: in-text, marginal and pop up.  \textcite{abuseileek2008} finds that out of these three the marginal presentation category performs the best in both vocabulary retention and reading comprehension, while \textcite{marefat2016} finds that pop-ups are more effective than marginal for reading comprehension. As \textcite{chen2016} says, there has not been sufficient research into gloss location for valid conclusions to be drawn. 

\subsection{Gloss Taxonomy}
Gloss taxonomy is the content of the gloss.

\textcite{gettys2001} finds that dictionary style translations perform better than sentence level translations, suggesting that it would be better to unconjugate the word and then translate it, providing grammatical information of the conjugation method. 

There is a large body of research that suggests that providing multimedia glosses is more effective than providing text only ones. \autocite{yoshii2006, kost1999}. This is harder to implement  as it would the application to have a better understanding of the word's context to provided an accurate image of the word. 
