\documentclass{article}
\usepackage[utf8]{inputenc}


\begin{document}
\begin{center}
    {\large Part III Individual Project Brief}\\
    {\Huge Machine Assisted Reading of Texts in German}\\
    Student: Steffan Padel\\
    Supervisor: Adam Prugel-Bennett\\
\end{center}

\section{Problem}

There have been a number of e-learning tools developed to assist people with the learning of foreign languages. Very few of these tools provided real-time assistance when a person is trying to read a text and of the ones that do, very few tailor this assistance to the user's current ability with the language.

\section{Goals}

To build a tool that when given a text, such as a news article or opinion piece in German. The tool will then take the text, analyse it and then proceed to identify and highlight the words that are either unknown to the user or difficult for them to understand.

The user should then be able to click on this word and see both a translation of this word as well as any grammatical rules associated with the word, for example the root word and how the word was mutated for verbs or the gender of the word for nouns.

The tool will also be responsive, adjusting which words it highlights based on which words the user has clicked while reading previous articles.

\section{Scope}

I will be building a web application that fills the above goals. It will allow users to log on, then allow them to submit articles from German language news websites, the tool will then scan the article, and return a version of it with the words that are unknown or unfamiliar to the user highlighted and clickable.

The application will start by asking the user for their skill level on a scale of one to ten. It will then adjust  its understanding of the user's skill based on which words the user needs help with.

The German language has been chosen for this as it is a language that I am familiar enough with in order to write this tool.

\end{document}